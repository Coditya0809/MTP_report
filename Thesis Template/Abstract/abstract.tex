
In today's technology, high-speed links play an important role, enabling 
faster, cheaper, and more reliable data communication. Data converters, present in some form in almost all modern high-speed links, are a key to performing equalization - the process of compensating bandwidth limitations of the communication channel. In particular, high-speed ADCs at the receiver front ends can enable easily-scalable digital implementations of various equalization schemes, such as feed-forward equalization (FFE), decision-feedback equalization (DFE), and even maximum-likelihood sequence estimation (MLSE). However, power limitations of on-chip high-speed link receivers make front-end ADC design very challenging. Therefore, in this work we design a time-interleaved ADC which can be used in the Receiver front end of a High-speed SerDes Link. This leads us to the idea of heavily interleaving very simple and efficient ADCs to obtain high aggregate data conversion rates with low resolution ADCs working at comparatively mediocre data conversion rates. In this work specifically, a 16-way time-interleaved SAR ADC has been targeted to work with a conversion rate of 16 Gbps on SerDes links. This means that each sub-ADC's targeted conversion rate will be relaxed to only 1 Gbps. Various design elements of the SAR ADC have been designed and simulated separately. All these simulations have been done in Cadence Virtuoso tool. These circuits are implemented using the USMC 65\,nm 1.2\,V CMOS technology. The statistical performance of the SerDes Link with the inclusion of ADC on the receiver side is also being studied and simulated. The time-domain simulations of the SerDes Links are being done in MATLAB R2022a.

\newpage
%\null\newpage